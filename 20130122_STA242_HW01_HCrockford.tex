\documentclass[12pt]{article}
\usepackage{graphicx}
\usepackage{listings}
\title{STA242 - H01}
\author{Hugh Crockford}
\date{January 20,2013}
\begin{document}
	\maketitle
	\tableofcontents
	\clearpage
	\section{Late Flights - by airline }
	\subsection{Q: What airports are the worst for on time performance?}
		Examining which airports are worst for on time performance may be useful to see if there are any airports or areas which are consistently poor performers.
		\begin{figure}[h!b]
			\centering
			\includegraphics[scale=0.5]{pcl_airport.jpg}
			\caption{Percent flights that are late ( more than 30 min) by airport. }
		\end{figure}
	\newpage
	\subsection{Q: Is there a relationship between airport location and on time performance?}
		\begin{figure}[h!]
			\centering
			\includegraphics[scale=0.5]{latemap.jpg}
			\caption{Map showing location of airport}
		\end{figure}
		The above figure shows the regional hubs(Ohare, LAX etc)  are the worst for on time performance, which is to be expected as they are most likely to suffer the effects of concertina like accumulation of delays.
	\newpage
	\subsection{Q: Which airlines are the worst for on time performance?}
	\begin{figure}[h!]
		\centering
		\includegraphics[scale=0.5]{lateCar.jpg}
		\caption{Late flights ( +- >30 min ) by carrier}
	\end{figure}
	The Above figure shows a ranking of airlines by on time percentage.
	An investigation of the effect of lateness threshold on on time performance follows in section 3.
 	\newpage 
	\section{Late flights - by time of year}
		\subsection{Q: Is there a pattern of delays throughout the year?}
			An investigation into the seasonal pattern of delays, and the reason for the delays follows.
		 	\begin{figure}[h!]
				\centering
				\includegraphics[scale=0.5]{delaydate.jpg}
				\caption{Percent of flight delayed, by time of year}
			 \end{figure}
			The figure above does not display any strong seasonal tendancies, although this analysis may be getting confounded by delay reason, i.e there actually is an increase in weather delays in winter, but there is a corresponding increase in other reasons for delays during summer which offsets this.
			\clearpage
		\subsection{Q: Is there more weather delays in winter?}
		\begin{figure}[h!]
			\centering
			\includegraphics[scale=0.5]{weather.jpg}
			\caption{Patterns of delay reason - more weather delays in winter?}
		\end{figure}
		Increasing the resolution of delay reason and subsetting for data encoded reasons(inbound, outbound, late plane, weather, carrier) revealed there actually was an increase in weather delays during winter, as well as carrier delay (maybe some carrier delays are being miscalssified by carrier or data collecter as carrier delays when they are infact due to weather?)
		\newpage
	\section{Late flights - effect of cutoff value.}
		\subsection{Q: Does late time cutoff affect percent planes late}
		The relationship between cutoff for flight delays and percent delayed by various groups was examined.\\
		This investigation may reveal some airlines/airports which are often late by only a little and others that are rarely late, but when they are planes miss their scheduled departure/arrival time by a large margin.\\
		\begin{figure}[h!]
			\centering
			\includegraphics[scale=0.5]{lateness.jpg}
			\caption{Percent of planes late by various cutoffs}
		\end{figure}
		The above figure shows the expected relationship between late time cutoff (both arriving and departing), and the percent of late planes, i.e as the late time cutoff is increased, there are less planes classified as late.
	\newpage
		\subsection{Q: Do different airlines/airports perform better under different late cutoff points}
		\begin{figure}[h!]
			\centering
			\includegraphics[scale=0.5]{ltcarrier.jpg}
			\caption{Plot of lateness by carrier, for different late cutoffs.}
		\end{figure}
		The above figure shows there is a relationship between different airlines and the effects of lateness cutoff. \\
		MQ performs consitently well over all measures of lateness, exceeding its peers at every cutoff.
		\\HA on the other hand performs well at a cutoff of 60 minutes (ranked 6th of 14), however performs worse than its peers as the late time cutoff is increased (at late = 360 (6hours), HA has 7 times the late planes than the average of all airlines). This suggests that if HA experiences a delay, it is a signifigant delay. (But I guess if you're going to be stuck somewhere, Honolulu isn't a bad place to be)
		\\The below figures shows there is a relationship between the destination and origin airport, and the effects of lateness cutoff		
	\newpage
		\begin{figure}[h!]
			\centering
			\includegraphics[scale=0.5]{ltorig.jpg}
			\caption{Plot of lateness by origin airport, for different late cutoffs.}
			\includegraphics[scale=0.5]{ltdest.jpg}
			\caption{Plot of lateness by destination airport, for different late cutoffs.}
		\end{figure}
	\clearpage
	\section{CODE}
	\lstinputlisting[breaklines=true]{``H01\_airline.R''}
\end{document}
